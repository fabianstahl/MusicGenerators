%%%%%%%%%%%%%%%%%%%%%%%%%%%%%%%%%%%%%%%%%%%%%%%%%%%%%%%%%%%%%%%%%%%%%%%%%%%%%%%%
% CVMR Poster Template based on:
% Dreuw & Deselaer's Poster Version 1.0 (11/04/13)
% http://www-i6.informatik.rwth-aachen.de/~dreuw/latexbeamerposter.php
%%%%%%%%%%%%%%%%%%%%%%%%%%%%%%%%%%%%%%%%%%%%%%%%%%%%%%%%%%%%%%%%%%%%%%%%%%%%%%%%

\documentclass[final,hyperref={pdfpagelabels=false}]{beamer}
\usepackage[orientation=portrait,size=a0,scale=1.4]{beamerposter} % Use the beamerposter package for laying out the poster with a portrait orientation and an a0 paper size

\usetheme{cvmr}

\usepackage[english]{babel} % English language/hyphenation
\usepackage{amsmath,amsthm,amssymb,latexsym} % For including math equations, theorems, symbols, etc
\usepackage{enumitem,xcolor}
\usepackage[scaled]{helvet}
\usepackage[percent]{overpic}
\usepackage{microtype}
\setlength{\emergencystretch}{1em}
\usepackage{booktabs} % Top and bottom rules for tables
\usepackage{subfig}
\usepackage[justification=justified]{caption}
\usepackage{adjustbox}
\usepackage[binary-units=true]{siunitx}
\usepackage{tikz}
\usetikzlibrary{shapes,arrows,calc,backgrounds,positioning,fit}
\usepackage[acronym,xindy]{glossaries}
\newacronym{cvmr}{CVMR}{Computer Vision and Mixed Reality}
\newacronym{hsrm}{HSRM}{High-Speed and Robust Monocular}

\newacronym{icarus}{Icarus}{Infrastructure for Compact Aerial Robots Under Supervision}

\newacronym{mav}{MAV}{Micro Aerial Vehicle}
\newacronym{rmse}{RMSE}{Root-Mean-Square Error}
\newacronym{ros}{ROS}{Robot Operating System}
\newacronym{sar}{SAR}{Search and Rescue}
\newacronym{slam}{SLAM}{Simultaneous Localization and Mapping}
\newacronym{ugv}{UGV}{Unmanned Ground Vehicle}


\newcommand{\blocktextwidth}{0.93\textwidth}

% graphics path
\graphicspath{{./figures/}}

% input path
\makeatletter
\providecommand*{\input@path}{}
\edef\input@path{{./figures/}\input@path}% prepend
\makeatother

%-------------------------------------------------------------------------------

\newcommand{\icarus}{\textit{ICARUS}}

%---TITLE-----------------------------------------------------------------------

\title{
A Low-Cost Mobile Infrastructure\\[0.5ex] for Compact Aerial Robots Under Supervision
} % Poster title

\author{%
Marc Lieser \and Henning Tjaden \and Robert Brylka \and Lasse L{\"o}ffler \and Ulrich Schwanecke
} % Author(s)

\institute{%
Computer Vision \& Mixed Reality Group, Hochschule RheinMain University of Applied Sciences, Wiesbaden
} % Institution(s)

%---FOOTER----------------------------------------------------------------------

\newcommand{\leftfoot}{http://cvmr.mi.hs-rm.de/icarus} % Left footer text
\newcommand{\rightfoot}{marc.lieser@hs-rm.de} % Right footer text

%-------------------------------------------------------------------------------

\begin{document}

\begin{frame}[t] % The whole poster is enclosed in one beamer frame

\begin{columns}[t] % The whole poster consists of three major columns, each of which can be subdivided further with another \begin{columns} block - the [t] argument aligns each column's content to the top

%%%%%%%%%%%%%%%%%%%%%%%%%%%%%%%%%%%%%%%%%%%%%%%%%%%%%%%%%%%%%%%%%%%%%%%%%%%%%%%%
%        LEFT  COLUMN        %%%%%%%%%%%%%%%%%%%%%%%%%%%%%%%%%%%%%%%%%%%%%%%%%%%
%%%%%%%%%%%%%%%%%%%%%%%%%%%%%%%%%%%%%%%%%%%%%%%%%%%%%%%%%%%%%%%%%%%%%%%%%%%%%%%%

\begin{column}{.31\textwidth} % The first column

%-------------------------------------------------------------------------------

{
\setbeamercolor{block title}{fg=myfg,bg=white} % Change the block title color
\setbeamercolor{block body}{fg=myfg,bg=white} % Change the block title color

%---INTRODUCTION----------------------------------------------------------------

\vspace{1.11em}

\begin{block}{Introduction}
\begin{minipage}[]{\blocktextwidth}
The availability of affordable \glspl{mav} opens up a whole new field of civil applications.
We present an \emph{Infrastructure for Compact Aerial Robots Under Supervision} (\icarus{})~\cite{lieser2017} that realizes a scalable low-cost testbed for research in the area of \glspl{mav} starting at about \$100.
It combines hardware and software for tracking and computer-based control of multiple quadrotors.
In combination with the usage of lightweight miniature off-the-shelf quadrotors our system provides a testbed that virtually can be used anywhere without the need of elaborate safety measures.
We give an overview of the entire system, provide some implementation details as well as an evaluation and depict different applications based on our infrastructure such as an \gls{ugv} which in cooperation with a \gls{mav} can be utilized in \gls{sar} operations
and a multi-user interaction scenario with several \glspl{mav}.
\end{minipage}
\end{block}
}

%---COMPONENTS------------------------------------------------------------------
            
\begin{block}{Components}
\begin{minipage}[]{\blocktextwidth}
Our low-cost quadrotor testbed consists of miniature off-the-shelf quadrotors, multiple radio remote controls and an infrared-based monocular tracking system~\cite{tjaden2014}.
It consequently utilizes low-cost off-the-shelf hardware and thus a minimum setup can be realized for under \$100 on a single-board computer.
\end{minipage}
\end{block}

%---ARCHITECTURE----------------------------------------------------------------

\begin{block}{Software Architecture}
\begin{minipage}[]{\blocktextwidth}
High modularity and platform independency gain the scalability and flexibility required by our infrastructure. 
A publish-subscribe messaging pattern ensures loose coupling and a dynamic network topology which allows the use of a dedicated tracking computer and to control the infrastructure using a mobile phone.
\end{minipage}

\begin{figure}[htbp]
\centering
\resizebox{0.9\columnwidth}{!}{\input{software_architecture}}
\end{figure}

\end{block}

\vspace{-0.7em}

%---REFERENCES------------------------------------------------------------------

{
\setbeamercolor{block title}{fg=myfg,bg=white} % Change the block title color
\setbeamercolor{block body}{fg=myfg,bg=white} % Change the block title color

\begin{block}{References}
\vspace{-.5\baselineskip}
\bibliographystyle{unsrt}
\scriptsize{\bibliography{./literatur}}
\end{block}
}

%-------------------------------------------------------------------------------

\end{column} % End of the first column

%%%%%%%%%%%%%%%%%%%%%%%%%%%%%%%%%%%%%%%%%%%%%%%%%%%%%%%%%%%%%%%%%%%%%%%%%%%%%%%%
%%%%%%%%%%%%%%%%%%%%%%%%%%%           SPANNED MIDDLE AND RIGHT COLUM           %
%%%%%%%%%%%%%%%%%%%%%%%%%%%%%%%%%%%%%%%%%%%%%%%%%%%%%%%%%%%%%%%%%%%%%%%%%%%%%%%%

\begin{column}{0.6375\textwidth} % The second column

\begin{figure}
\centering
\input{architecture_overpic.tex}
\caption*{%
Overview of \icarus{}: After pose estimation of the individual quadrotors based on optical tracking the control variables are determined and sent via different radio remote control devices to the individual quadrotors.
}
\end{figure}

\vspace{-0.55em}

\begin{columns}[T]

%%%%%%%%%%%%%%%%%%%%%%%%%%%%%%%%%%%%%%%%%%%%%%%%%%%%%%%%%%%%%%%%%%%%%%%%%%%%%%%%
%%%%%%%%%%%%%%%%%%%%%%%%%        MIDDLE  COLUMN        %%%%%%%%%%%%%%%%%%%%%%%%%
%%%%%%%%%%%%%%%%%%%%%%%%%%%%%%%%%%%%%%%%%%%%%%%%%%%%%%%%%%%%%%%%%%%%%%%%%%%%%%%%

\begin{column}{0.486\textwidth}

%---EVALUATION------------------------------------------------------------------

\begin{block}{Evaluation}
\begin{minipage}[]{\blocktextwidth}
In the \textbf{skywriting} experiment, a quadrotor followed the trajectory of the letters CVMR with a total width of \SI{1000}{\milli\meter} and a height of \SI{292}{\milli\meter} using a hover controller.
The baseline of the lettering was at a height of \SI{1000}{\milli\meter}.
The \gls{rmse} of this experiment is \SI{22.2}{\milli\meter}.
\end{minipage}

\vspace{0.8em}

\begin{figure}
\centering\tiny
\adjustbox{width=0.9\columnwidth,trim=0em 1em 3em 2em}{\input{cvmr}}
\end{figure}

\vspace{1em}

\begin{minipage}[]{\blocktextwidth}
A flight in which the quadrotor was instructed to \textbf{hover} at the target height of \SI{1300}{\milli\meter} for \SI{60}{\second} was carried out in a single camera environment.
The \gls{rmse} of this experiment is \SI{7.6}{\milli\meter}, the mean and standard deviations are $\bar{\mathbf{t}} = (0.6\pm4.4, -0.2\pm5.9, -2.3\pm4.1)$, and the maximum deviations are $\mathbf{d}_{\textnormal{max}} = (12.6, 18.3, 20.8)$.
\end{minipage}

\vspace{0.1em}

\begin{figure}
\centering\tiny
\adjustbox{width=0.85\columnwidth,trim=-1em 2em 0em 1em}{\input{hover}}
\end{figure}

\end{block}

%-------------------------------------------------------------------------------

\end{column}

\hfill

%%%%%%%%%%%%%%%%%%%%%%%%%%%%%%%%%%%%%%%%%%%%%%%%%%%%%%%%%%%%%%%%%%%%%%%%%%%%%%%%
%%%%%%%%%%%%%%%%%%%%%%%%%%%%%%%%%%%%%%%%%%%%%%%%%%%%%%      RIGHT  COLUMN      %
%%%%%%%%%%%%%%%%%%%%%%%%%%%%%%%%%%%%%%%%%%%%%%%%%%%%%%%%%%%%%%%%%%%%%%%%%%%%%%%%

\begin{column}{0.486\textwidth}

%---APPLICATIONS----------------------------------------------------------------

\begin{block}{Applications}
\begin{minipage}[]{\blocktextwidth}
As proof of concept we realized different applications based on \icarus{}:
\textbf{Light paintings} created by the quadrotors' trajectories;
expanding the limited field of view of \glspl{ugv} with a quadrotor for the use in \textbf{\gls{sar}} applications;
experimental tactile and gesture-based \textbf{human-quadrotor interaction} scenarios.
\end{minipage}
\end{block}

\begin{figure}
\shortstack{%
\includegraphics[height=92mm]{lp_cvmr.jpg}\\[2mm]
\includegraphics[height=92mm]{lp_circles.jpg}
}
\hfill
\includegraphics[height=187mm]{figures/flyper.jpg}
\caption*{Light paintings of quadrotor evaluation trajectories (left). A \gls{mav} controlled by an experimental \gls{ugv} (right).}
\end{figure}

%---FUTURE-WORK-----------------------------------------------------------------

\vspace{-0.8em}

\begin{block}{Future Work}
\begin{minipage}[]{\blocktextwidth}
\begin{itemize}[leftmargin=*,labelindent=0pt,label={\color{black!40}$\bullet$}]
\item{Escape the limited indoor environment utilizing visual SLAM algorithms}
\item{Solve model predictive control (MPC) quadratic problems to generate and track feasible smooth trajectories}
\item{Explore new (tactile) human-quadrotor interaction methods}
\item{Open source parts of our implementation}
\item{Integrate Arduino-based RC approaches and quadrotor simulation into ROS}
\end{itemize}
\end{minipage}
\end{block}

%-------------------------------------------------------------------------------

\end{column}
\end{columns}

%-------------------------------------------------------------------------------

\end{column} % End of the second column
\end{columns} % End of all the columns in the poster

%-------------------------------------------------------------------------------

\begin{figure}[htp]
\centering
\begin{minipage}[t]{\columnwidth}{\footnotesize
\begin{overpic}[width=.1\columnwidth]{mimic/series_0000_initial.jpg} {\put (72,9) {1}}\end{overpic}%
\begin{overpic}[width=.1\columnwidth]{mimic/series_0001_select.jpg}  {\put (72,9) {2}}\end{overpic}%
\begin{overpic}[width=.1\columnwidth]{mimic/series_0002_takeoff.jpg} {\put (72,9) {3}}\end{overpic}%
\begin{overpic}[width=.1\columnwidth]{mimic/series_0003_hover_1.jpg} {\put (72,9) {4}}\end{overpic}%
\begin{overpic}[width=.1\columnwidth]{mimic/series_0004_hover_2.jpg} {\put (72,9) {5}}\end{overpic}%
\begin{overpic}[width=.1\columnwidth]{mimic/series_0005_hover_3.jpg} {\put (72,9) {6}}\end{overpic}%
\begin{overpic}[width=.1\columnwidth]{mimic/series_0006_hover_4.jpg} {\put (72,9) {7}}\end{overpic}%
\begin{overpic}[width=.1\columnwidth]{mimic/series_0007_unselect.jpg}{\put (72,9) {8}}\end{overpic}%
\begin{overpic}[width=.1\columnwidth]{mimic/series_0008_land.jpg}    {\put (72,9) {9}}\end{overpic}%
\begin{overpic}[width=.1\columnwidth]{mimic/series_0009_final.jpg}   {\put (72,9) {10}}\end{overpic}}%
\end{minipage}
\caption*{Gesture-based interaction: A user selects a quadrotor by pointing at it with the right arm~(2). Lifting the left arm instructs the quadrotor to take off~(3). While the left arm is up, the quadrotor imitates the motion of the user's right hand~(4--7). When lowering the left arm, the quadrotor may be selected by another user~(8). Bringing both hands together~(9) lets the quadrotor land~(10).}
\end{figure}

%-------------------------------------------------------------------------------

\end{frame} % End of the enclosing frame

\end{document}
